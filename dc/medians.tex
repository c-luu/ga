\documentclass{article}
\usepackage{amsmath}
\newtheorem{theorem}{Theorem}[section]
\newtheorem{property}{Property}[section]
\newtheorem{eg}{Example}
\newtheorem{corollary}{Corollary}[theorem]
\newtheorem{lemma}[theorem]{Lemma}
\newtheorem{remark}{Remark}

\begin{document}	
\section{Medians}
Given an unordered sequence, $A$, we want to find the $k$'th largest or smallest element without sorting $A$ in $O(n)$. 

\begin{remark}
\label{bc1}	
Medians is inspired by the QuickSort algorithm.
\end{remark}

\begin{eg}
\label{eg1}
\end{eg}
Let $A = [5,2,20,17,11,13,8,9,11]$ and $p = 11$.\\
If we use $p$ to partition $A$ into the subsequences $A_{<p}, A_{=p}, A_{>p}$ in $O(n)$, we get the following:\\

\begin{align*}
&A_{<p} = [5,2,8,9,11]\\
&A_{=p} = [11,11]\\
&A_{<p} = [20,17,13]
\end{align*}

We have a useful property from the subsequences:
\begin{align*}
&|A_{<p}| = 5\\
&|A_{=p}| = 2\\
&|A_{>p}| = 3
\end{align*}

S.t. if we copied those three subsequence to a new array: $$A'= A_{<p} + A_{=p} + A_{<p}$$ we have a property that $A'[0..|A_{<p}|] = A_{<p}$ and so on. Thus, if we wanted to find $k$ s.t. $k<5$, we know to look in $A_{<p}$. If we wanted to find $k$ s.t. $7 \leq k < |A'|$, we know to look in $A_{>p}$ since $A'[7..|A'|] = A_{>p}$ .

\begin{property}
\label{p1}
What if we want to find $k$ s.t. $|A_{<p}| < k \leq |A_{=p}|+|A_{<p}|$? E.g., in the case of ex. $\ref{eg1}$ above: 

$$5 < k \leq 2+5 = 5 < k \leq 7$$

We simply return any element in $A_{=p}$. A corollary is this property saves a recursive search.
\end{property}

\begin{remark}[Good Pivot via Random Picking]
We can find a good pivot in \textbf{expected} time of $O(n)$ since there is a $50\%$ chance of picking one, however this is NOT the same as saying this is the guaranteed worst-case running time of $O(n)$.
\end{remark}

\end{document}