\documentclass{article}
\usepackage[T1]{fontenc}
\usepackage{amssymb, amsmath, graphicx, subfigure, enumerate}
\usepackage{amsthm,alltt}
\usepackage[margin=1.25in]{geometry} %geometry (sets margin) and other useful packages
\usepackage{graphicx,ctable,booktabs}
\usepackage{mathtools}
\usepackage[boxed]{algorithm2e}
\usepackage{mathdots}
\usepackage{fancyhdr} %Fancy-header package to modify header/page numbering
\usepackage{cleveref}
\usepackage{hyperref}
\usepackage{algorithmic}
\usepackage{algorithm}


\setlength{\oddsidemargin}{.25in}
\setlength{\evensidemargin}{.25in}
\setlength{\textwidth}{6in}
\setlength{\topmargin}{-0.4in}
\setlength{\textheight}{8.5in}

\newcommand{\heading}[6]{
  \renewcommand{\thepage}{\arabic{page}} % used to be #1-\arabic{page}
  \noindent
  \begin{center}
  \framebox{
    \vbox{
      \hbox to 5.78in { \textbf{#2} \hfill #3 }
      \vspace{4mm}
      \hbox to 5.78in { {\Large \hfill #6  \hfill} }
      \vspace{2mm}
      \hbox to 5.78in { \textit{Instructor: #4 \hfill #5} }
    }
  }
  \end{center}
  \vspace*{4mm}
}

%Redefining sections as problems
\makeatletter
\newenvironment{problem}{\@startsection
       {section}
       {2}
       {-.2em}
       {-3.5ex plus -1ex minus -.2ex}
       {2.3ex plus .2ex}
       {\pagebreak[3]%forces pagebreak when space is small; use \eject for better results
       \large\bf\noindent{Problem }
       }
       }
       %{%\vspace{1ex}\begin{center} \rule{0.3\linewidth}{.3pt}\end{center}}
       %\begin{center}\large\bf \ldots\ldots\ldots\end{center}}
\makeatother


\newtheorem{theorem}{Theorem}[section]
\newtheorem{definition}[theorem]{Definition}
\newtheorem{remark}[theorem]{Remark}
\newtheorem{lemma}[theorem]{Lemma}
\newtheorem{corollary}[theorem]{Corollary}
\newtheorem{proposition}[theorem]{Proposition}
\newtheorem{claim}[theorem]{Claim}
\newtheorem{observation}[theorem]{Observation}
\newtheorem{fact}[theorem]{Fact}
\newtheorem{assumption}[theorem]{Assumption}

%\newenvironment{proof}{\noindent{\bf Proof:} \hspace*{1mm}}{
% \hspace*{\fill} $\Box$ }
%\newenvironment{proof_of}[1]{\noindent {\bf Proof of #1:}
% \hspace*{1mm}}{\hspace*{\fill} $\Box$ }
%\newenvironment{proof_claim}{\begin{quotation} \noindent}{
% \hspace*{\fill} $\diamond$ \end{quotation}}

\newcommand{\problemset}[3]{\heading{#1}{\classname}{#2}{\instructor}{#3}{}} % Don't change this line
%%%%%%%%%%%%%%%%%%%%%%%%%% Change this stuff below, don't change the line above this one
\newcommand{\problemsetnum}{}            % problem set number
\newcommand{\duedate}{} % problem set deadline
\newcommand{\studentname}{}      % name of student (i.e., you)
\newcommand{\classname}{}
%%%%%%%%%%%%%%%%%%%%%%%%%%

\pagestyle{fancy}
%\addtolength{\headwidth}{\marginparsep} %these change header-rule width
%\addtolength{\headwidth}{\marginparwidth}
\lhead{\classname} %Problem \thesection}
\chead{}
\rhead{\thepage}
%\lfoot{\small\scshape \classname}
\cfoot{}
%\rfoot{\footnotesize PS \#\problemsetnum}
\renewcommand{\headrulewidth}{.3pt}
\renewcommand{\footrulewidth}{.3pt}
\setlength\voffset{-0.25in}
\setlength\textheight{648pt}


\newcommand{\sit}{\shortintertext}
\newcommand\deq{\mathrel{\overset{\makebox[0pt]{\mbox{\normalfont\tiny\sffamily def}}}{=}}}
\newcommand{\ones}{\mathbbm{1}}
\newcommand{\e}{\vec{e}}
\newcommand{\tr}{\text{tr}}
\newcommand{\bs}{\boldsymbol}
\mathchardef\mhyphen="2D
\newcommand{\C}{\mathbb{C}}
\newcommand{\R}{\mathbb{R}}
\newcommand{\II}{\mathcal{I}}
\newcommand{\FF}{\mathcal{F}}
\newcommand{\X}{\mathcal{X}}
\newcommand{\Y}{\mathcal{Y}}
\newcommand{\ra}{\rightarrow}
\newcommand{\Ra}{\Rightarrow}
\newcommand{\PP}{\mathbb{P}}
\newcommand{\sse}{\subseteq}
\newcommand{\eps}{\epsilon}
\newcommand{\N}{\mathcal{N}}
\newcommand{\poly}{\textup{poly}}

\newcommand{\dom}{\textup{dom}}

\renewcommand{\thesubsection}{\thesection.\roman{subsection}}


% auto sized delimiters
\newcommand{\Br}[1]{\left\{#1\right\}}
\newcommand{\br}[1]{\left[#1\right]}
\newcommand{\pr}[1]{\left(#1\right)}
\newcommand{\ceil}[1]{\left\lceil#1\right\rceil}
\newcommand{\floor}[1]{\left\lfloor#1\right\rfloor}
\newcommand{\abs}[1]{\left|#1\right|}
\newcommand{\sgn}{\textup{sgn}}

%default delimiter for Pr and E
\DeclarePairedDelimiter{\defaultDelim}{[}{]}

\DeclareMathOperator{\capPr}{Pr}
\renewcommand{\Pr}[2][]{\capPr_{#1}\defaultDelim*{#2}}
\DeclareMathOperator{\capE}{E}
\newcommand{\E}[2][]{\capE_{#1}\defaultDelim*{#2}}
\DeclareMathOperator{\capVar}{Var}
\newcommand{\Var}[2][]{\capVar_{#1}\defaultDelim*{#2}}

%\DeclareMathOperator*{}{} puts subscript below


%%%%%%%%%%%%%%%%%%%%%%%%%%%%%%%%%%%%%%%%%%%%%%%%%
\begin{document}
% \problemset{\problemsetnum}{\duedate}{\studentname}
\begin{problem} {(path of maximal capacity)}	
Consider a flow network ($G = (V, E), s, t, \{c_e\}$). The capacity of a path from $s$ to $t$ is defined as the minimum capacity of its edges. Give an algorithm to find a path from $s$ to $t$ of maximal capacity. Describe your algorithm with words (no pseudocode) and analyze its running time.
\end{problem}\\

\begin{definition}{$P[v]: v \in V$}
The array maintained by $Dijkstra$'s algorithm returning the parent of some $v \in V$.
\end{definition}

\begin{remark}
\label{q}
We use a heap-based priority queue, $Q$, pops vertices in descending order of their $max\_c$ priority values. I.e., if $v$ is popped from $Q$, $max\_c[v]$ is the maximum value in $Q - \{v\}$. W.l.o.g., \ref{alg:rel_new} implicitly pushes $v$ to $Q$ with priority $max\_c[v]$ if $(u,v)$ is relaxed.
\end{remark}

\begin{algorithm}
\caption{$relax(u,v,w)$}
\label{alg:rel}
\begin{algorithmic}
\IF{$d[u] > d[u]+w(u,v)$}
	\STATE $d[v] \gets d[u]+w(u,v)$
	\STATE $P[v] \gets u$
\ENDIF
\end{algorithmic}
\end{algorithm}

\begin{algorithm}
\caption{$relax'(u,v)$}
\label{alg:rel_new}
\begin{algorithmic}
\IF{$min(max\_c[u], c_{(u,v)}) > max\_c[v]$}
	\STATE $max\_c[v] \gets min(max\_c[u], c_{(u,v)})$
	\STATE $P[v] \gets u$
\ENDIF
\end{algorithmic}
\end{algorithm}

We can accomplish this by modifying $Dijkstra$'s (from $CLRS$) algorithm for computing single-shortest paths. The key modification is modifying the original $relax$ algorithm to \ref{alg:rel_new}, and returning a path using $P[t]$ starting at $t$ and back-tracking the parent links until $s$ is reached. 

\begin{proof}
We alter the definition of \textit{shortest path of some} $s \rightsquigarrow v$ to mean \textit{the maximum, minimum capacity path of some} $s \rightsquigarrow v$. 

For initial conditions, assume $\forall v \in V - \{s\}, v \in Q$ and $max\_c[v] = -\infty$, and set $S$ contains $s$, with $max\_c[s] = \infty$. In each iteration, $S$ holds the vertex popped from $Q$. 

The inductive hypothesis declares that $S$ will contain vertices with the correct shortest paths from $s$ calculated, and $Q$ has vertices where we have not calculated their shortest paths. The base case is trivial, since in the beginning it only includes $s$, and by our initial conditions its shortest path is to itself at $\infty$.

For the inductive step, we pop a vertex $u$ from $Q$ and add it to $S$. By \ref{q}, we know $u$ has the shortest path from $s$, or maximum $max\_c$ in $Q$. All $s-u$ paths must have capacities $\leq max\_c[u]$ since the remaining reachable paths must go through some vertex in $Q$. For non $s-u$ paths: $u$ was popped before all such vertices in $Q$ due to its priority, we know $\forall v \in Q, \, max\_c[v] \leq max\_c[u]$.

Therefore, $max\_c[u]$ is optimal, by induction the shortest path will be correctly computed upon termination.

Since \ref{alg:rel_new} is the core change in this modified $Dijkstra$'s algorithm, with an implicit priority increasing push to $Q$ as in the original algorithm, and it's a heap-based priority queue by \ref{q}, the algorithm can be run in $O((m+n)logn)$ time.

\end{proof}
 
\end{document}