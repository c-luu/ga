\documentclass{article}
\usepackage[parfill]{parskip}
\usepackage[T1]{fontenc}
\usepackage{amssymb, amsmath, graphicx, subfigure, enumerate}
\usepackage{amsthm,alltt}
\usepackage[margin=1.25in]{geometry} %geometry (sets margin) and other useful packages
\usepackage{graphicx,ctable,booktabs}
\usepackage{mathtools}
\usepackage[boxed]{algorithm2e}
\usepackage{mathdots}
\usepackage{fancyhdr} %Fancy-header package to modify header/page numbering
\usepackage{hyperref}


\setlength{\oddsidemargin}{.25in}
\setlength{\evensidemargin}{.25in}
\setlength{\textwidth}{6in}
\setlength{\topmargin}{-0.4in}
\setlength{\textheight}{8.5in}

\newcommand{\heading}[6]{
  \renewcommand{\thepage}{\arabic{page}} % used to be #1-\arabic{page}
  \noindent
  \begin{center}
  \framebox{
    \vbox{
      \hbox to 5.78in { \textbf{#2} \hfill #3 }
      \vspace{4mm}
      \hbox to 5.78in { {\Large \hfill #6  \hfill} }
      \vspace{2mm}
      \hbox to 5.78in { \textit{Instructor: #4 \hfill #5} }
    }
  }
  \end{center}
  \vspace*{4mm}
}

%Redefining sections as problems
\makeatletter
\newenvironment{problem}{\@startsection
       {section}
       {2}
       {-.2em}
       {-3.5ex plus -1ex minus -.2ex}
       {2.3ex plus .2ex}
       {\pagebreak[3]%forces pagebreak when space is small; use \eject for better results
       \large\bf\noindent{Problem }
       }
       }
       %{%\vspace{1ex}\begin{center} \rule{0.3\linewidth}{.3pt}\end{center}}
       %\begin{center}\large\bf \ldots\ldots\ldots\end{center}}
\makeatother


\newtheorem{theorem}{Theorem}[section]
\newtheorem{definition}[theorem]{Definition}
\newtheorem{remark}[theorem]{Remark}
\newtheorem{lemma}[theorem]{Lemma}
\newtheorem{corollary}[theorem]{Corollary}
\newtheorem{proposition}[theorem]{Proposition}
\newtheorem{claim}[theorem]{Claim}
\newtheorem{observation}[theorem]{Observation}
\newtheorem{fact}[theorem]{Fact}
\newtheorem{assumption}[theorem]{Assumption}

%\newenvironment{proof}{\noindent{\bf Proof:} \hspace*{1mm}}{
% \hspace*{\fill} $\Box$ }
%\newenvironment{proof_of}[1]{\noindent {\bf Proof of #1:}
% \hspace*{1mm}}{\hspace*{\fill} $\Box$ }
%\newenvironment{proof_claim}{\begin{quotation} \noindent}{
% \hspace*{\fill} $\diamond$ \end{quotation}}

\newcommand{\problemset}[3]{\heading{#1}{\classname}{#2}{\instructor}{#3}{}} % Don't change this line
%%%%%%%%%%%%%%%%%%%%%%%%%% Change this stuff below, don't change the line above this one
\newcommand{\problemsetnum}{}            % problem set number
\newcommand{\duedate}{} % problem set deadline
\newcommand{\studentname}{}      % name of student (i.e., you)
\newcommand{\classname}{}
%%%%%%%%%%%%%%%%%%%%%%%%%%

\pagestyle{fancy}
%\addtolength{\headwidth}{\marginparsep} %these change header-rule width
%\addtolength{\headwidth}{\marginparwidth}
\lhead{\classname} %Problem \thesection}
\chead{}
\rhead{\thepage}
%\lfoot{\small\scshape \classname}
\cfoot{}
%\rfoot{\footnotesize PS \#\problemsetnum}
\renewcommand{\headrulewidth}{.3pt}
\renewcommand{\footrulewidth}{.3pt}
\setlength\voffset{-0.25in}
\setlength\textheight{648pt}


\newcommand{\sit}{\shortintertext}
\newcommand\deq{\mathrel{\overset{\makebox[0pt]{\mbox{\normalfont\tiny\sffamily def}}}{=}}}
\newcommand{\ones}{\mathbbm{1}}
\newcommand{\e}{\vec{e}}
\newcommand{\tr}{\text{tr}}
\newcommand{\bs}{\boldsymbol}
\mathchardef\mhyphen="2D
\newcommand{\C}{\mathbb{C}}
\newcommand{\R}{\mathbb{R}}
\newcommand{\II}{\mathcal{I}}
\newcommand{\FF}{\mathcal{F}}
\newcommand{\X}{\mathcal{X}}
\newcommand{\Y}{\mathcal{Y}}
\newcommand{\ra}{\rightarrow}
\newcommand{\Ra}{\Rightarrow}
\newcommand{\PP}{\mathbb{P}}
\newcommand{\sse}{\subseteq}
\newcommand{\eps}{\epsilon}
\newcommand{\N}{\mathcal{N}}
\newcommand{\poly}{\textup{poly}}

\newcommand{\dom}{\textup{dom}}

\renewcommand{\thesubsection}{\thesection.\roman{subsection}}


% auto sized delimiters
\newcommand{\Br}[1]{\left\{#1\right\}}
\newcommand{\br}[1]{\left[#1\right]}
\newcommand{\pr}[1]{\left(#1\right)}
\newcommand{\ceil}[1]{\left\lceil#1\right\rceil}
\newcommand{\floor}[1]{\left\lfloor#1\right\rfloor}
\newcommand{\abs}[1]{\left|#1\right|}
\newcommand{\sgn}{\textup{sgn}}

%default delimiter for Pr and E
\DeclarePairedDelimiter{\defaultDelim}{[}{]}

\DeclareMathOperator{\capPr}{Pr}
\renewcommand{\Pr}[2][]{\capPr_{#1}\defaultDelim*{#2}}
\DeclareMathOperator{\capE}{E}
\newcommand{\E}[2][]{\capE_{#1}\defaultDelim*{#2}}
\DeclareMathOperator{\capVar}{Var}
\newcommand{\Var}[2][]{\capVar_{#1}\defaultDelim*{#2}}

%\DeclareMathOperator*{}{} puts subscript below


%%%%%%%%%%%%%%%%%%%%%%%%%%%%%%%%%%%%%%%%%%%%%%%%%
\begin{document}
% \problemset{\problemsetnum}{\duedate}{\studentname}
\begin{problem} {(New maximal flow)}	
	You are given a directed graph $G = (V,E)$ with positive integer capacities $\{c_e\}_{e\in E}$ on each edge, a
	source $s$ and a sink $t$. You are also given an integer value maximal flow $\{f_e\}_{e\in E}$. Suppose we pick
	a specific edge $e$ and increase its capacity by 1. 
		
	Describe an algorithm to find a maximum flow on
	the new network. Your algorithm should run in linear time $O(n + m)$, where $n$ is the number of
	vertices and $m$ is the number of edges of G. Explain why your algorithm is correct (no pseudocode)
	and analyze its running time.
\end{problem}\\

\begin{definition}{$f^* = \{f_e\}_{e \in E}$}
\label{def:f_star}
	(This definition should be verified via \href{https://piazza.com/class/k52uzg6xjkl5xz?cid=616}{this thread.})
	
	The already computed maximum flow for all edges, $e \in E$.
\end{definition}

\begin{definition}{$\mathcal{P}$}
\label{prop:cal_p}	

	A path between $(s,t) \in G^f$ with some available capacity. I.e., $\mathcal{P}$ is an $f$-augmenting \textit{st-path} $\in G^f$. 
\end{definition}

\begin{definition}{$C$}
\label{def:c_rounds}

The size of max flow \textit{prior} to increasing the capacity of some $e$ by 1.
\end{definition}

\begin{lemma}
\label{lem:aug_path}
	For a flow, $f^*$: $$\nexists \mathcal{P} \in G^{f^*} \to f^* \text{is a max-flow}$$\\
\end{lemma}

\begin{claim}
Since we are given the maximal flow integer values for all $f_e \in E$, we could run $FordFulkerson$ to compute the new maximum flow in $O(m)$ in one round instead of $C = max flow$ rounds.
\end{claim}

\begin{proof}
	After increasing $e$'s capacity by 1, we construct $G^{f^*}$ using $f^*$, taking $O(n)$ time. We can do this while still preserving the pre-condition that available capacities must exist before constructing $G^{f^*}$ because we just increased some $e$'s capacity. We then check \ref{lem:aug_path} using $DFS$, taking $O(m)$ time. If $\mathcal{P}$ was found, we augment $f^*$ along this path in $O(n)$ time, otherwise return $f^*$.
	
	$C$ rounds have already been iterated to give us \ref{def:f_star}, so we only loop once more to generate the new max flow. Because capacity for some $e$ increases by 1, $C$ will increase by no more than 1 to arrive at the new max flow. $O(m) >O(n)$, since we assume $m = |E| \geq n-1 = |V|-1|$. Thus we can compute the new max flow in $O(m*1) = O(m)$ time.
\end{proof}

\begin{remark}
	$O(n)$ and $O(|E|)$ are not equivalent; I think you probably meant $O(m)$. To know how long it takes to loop through the edges depends how exactly the graph is stored, and we use adjacency list format, so there is no direct way to loop through the edges without looping through the vertices. This is the same reason that DFS is $O(n+m)$ even though intuitively it might only need to be $O(m)$ since you're really just exploring every edge. 
\end{remark}
 
\end{document}