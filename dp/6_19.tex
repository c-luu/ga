\documentclass{article}
\usepackage{amsmath}
\newtheorem{theorem}{Theorem}[section]
\newtheorem{corollary}{Corollary}[theorem]
\newtheorem{lemma}[theorem]{Lemma}
\newtheorem{remark}{Remark}

\begin{document}	
\section{Make Change with Unlimited Denominations and Knapsack of size k}
Given a value v, a set of non-zero denominations X, K(v, j) determines if $\exists$ a subset of X whose sum is v, with available denominations ending at $x_j$. x must be a set, rather than a multi-set as in 6.17 (repeated denominations allowed).

\begin{lemma} [Change]
\label{bc1}	
K(0,0) = true, an auxiliary lemma to help with the inductive hypothesis.
If $v - x_j = 0$ for some $x_j \in X$, $K(v-x_j, j-1) = true$. I.e., change has been made.
\end{lemma}

\subsection{Recurrence}
$\forall x_j \in X, j>0$:\\
\begin{equation}
K(v,j)=			
\begin{cases}
v = 0 \to true.\\	
\underset{x_j \leq v}{\bigvee} K(v-x_j, j-1), K(v, j-1).\\
false \text{, otherwise.}
\end{cases}
\end{equation}
The recurrence is similar to 6.17, but we add a choice $K(v,j-1)$, which is a recursive call to see if change can be made for v with $X - \{x_j\}$ denominations.
\end{document}