\documentclass{article}
\usepackage{amsmath}
\newtheorem{theorem}{Theorem}[section]
\newtheorem{corollary}{Corollary}[theorem]
\newtheorem{lemma}[theorem]{Lemma}
\newtheorem{remark}{Remark}

\begin{document}	
\section{Make Change with Unlimited Denominations and Knapsack of size k}
Given a value v, a set of non-zero denominations X, K(v) determines if $\exists$ a subset of X whose sum is v, with available denominations ending at $x_j$. The quantity of x must be $\leq$ k.

\begin{lemma} [Change]
\label{bc1}	
K(0,0) = true, an auxiliary lemma to help with the inductive hypothesis.
\end{lemma}

\subsection{Recurrence as a predicate}
$\forall x_j \in X$:\\
\begin{equation}
K(v,j)=			
\begin{cases}
v = 0 \land k \geq 0 \to true.\\	
k>0 \to \underset{x_j \leq v}{\bigvee} K(v-x_j, k-1).\\
false \text{, otherwise.}
\end{cases}
\end{equation}

\begin{remark}
While more elegant, this recurrence does not lend well to a straightforward implementation. As a hint, I think its runtime is $O(|X|kv)$.
\end{remark}

\subsection{Recurrence as a minima}
$\forall x_j \in X$:\\
\begin{equation}
K(v)=			
\underset{x_j \leq v}{min} \{1+K(v-x_j)\}.
\end{equation}

\begin{remark}
Thinking in terms of taking the minimum, with a base case: $T(0) = \infty$ allows the use of an array yielding $O(|X|v)$ runtime.
\end{remark}

\end{document}