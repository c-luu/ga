\documentclass{article}
\usepackage{amsmath}
\newtheorem{theorem}{Theorem}[section]
\newtheorem{corollary}{Corollary}[theorem]
\newtheorem{lemma}[theorem]{Lemma}
\newtheorem{remark}{Remark}

\begin{document}	
	\section{LCS}
	\subsection{Assumptions}
	Half-open intervals for sequence ranges, i.e. $[i,j)$.\\
	0-based indexing.
	\begin{theorem}
	Given two strings, $a$ and $b$, let $T(i, j)$ be a function, where $0 < i < |a|$ and $0 < j < |b|$, that returns the 
	maximum sub-string, z, of length $k$ in $a$ and $b$.
	\end{theorem}
	
	\begin{lemma} [Common sub-string]
	\label{zk}	
	$\exists z_k$ for some $i, j$ s.t. $a_{i, k+1}$ = $b_{j, k+1}$.
	\end{lemma}
	
	\begin{lemma} [Empty sub-strings]
	\label{cs}	
	If $a$ or $b$ are empty strings, then $T(i,j)=0$ because no common sub-string will be found.
	\end{lemma}
	
	\begin{lemma} [Termination]
	\label{term}	
	If $i=0 \vee j=0$, then $T(i,j)=0$ to terminate the search. This implies the $a$ and $b$ are left-padded with a junk character.
	\end{lemma}
	
	\begin{lemma} [Counting a match]
	\label{match}	
	T(i,j) = 1 +T(i-1,j-1) when  $a_i = a_j$, otherwise 0. Since a count has been recognized for the match at $i, j$, we add the count to the result of the previous step, T(i-1,j-1). I.e.,
	if there was a match at $i-1, j-1$, we may accumulate our count, preserving the sub-string property.
	\end{lemma}
	
	\begin{remark}
	An important note is lemma \ref{match} is crucial to maintain the sub-string accumulation. If we wanted the largest $k$ s.t. $z_k$ can be a non-contiguous sub-string, we would need
	an additional inductive case:
	$$\text{max}\{T(i-1,j), T(i,j-1)\}, \text{if $a_i \neq b_j$.}$$
	\end{remark}
	
	\subsection{Recurrence}
	 \begin{equation}
	T(i, j)=			
	\begin{cases}
	0, & \text{if $|a|=0 \vee |b|=0$ by \ref{cs}}.\\
	0, & \text{if $i=0 \vee j=0$ by \ref{term}}.\\	
	0, & \text{if } a_i \neq b_j \text{ by \ref{match}}.\\
	1+T(i-1,j-1), & \text{$a_i = b_j$ by \ref{match}}.				
	\end{cases}
	\end{equation}
\end{document}